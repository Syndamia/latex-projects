% Document must be built with XeLaTeX
% License and source at https://github.com/Syndamia/latex-projects
\documentclass[12pt,a4paper]{article}

\usepackage{pgfplots}  
\pgfplotsset{height=8cm, width=10cm}  

\usepackage{fontspec} % loaded by polyglossia, but included here for transparency 
\usepackage{polyglossia}
\setmainlanguage{russian} % alternative for lack of bulgarian support
\setotherlanguage{english}
\addto\captionsrussian{
  \renewcommand{\contentsname}{Съдържание}
  \renewcommand{\refname}{Източници}
}
\newfontfamily\russianfont{Times New Roman}

\bibliographystyle{plain}

\title{Статистическо изследване върху алгоритми за сортиране}
\author{Камен Димитров Младенов}
\date{23 Януари 2021}

\begin{document}

\maketitle
\tableofcontents

\newpage

\section{Същност и избор на темата}

Сортиращият алгоритъм е един от най-важните аспекти на програмирането в днешно време. 
Чрез него, дадени елементи могат да бъдат подредени в нарастваща (или намаляваща) последователност, което намира употреба във всичко от новинарски сайтове и социални медии, до видео игри и операционни системи.

Този тип алгоритъм бива изследван практически от началото на компютърната наука - най-ранните сортиращи алгоритми се забелязват през 1951, а анализиране на Bubble Sort, може би най-известния сортиращ алгоритъм, бива извършено още през 1957.\cite{wikipedia_sorting_algorithm_history}

В днешно време има десетки и стотици алгоритми, което дълбоко усложнява изборът на обикновения програмист. 
Затова тази статия се стреми да сравни главните им характеристики.

\section{Избор на алгоритми и разглеждани харакетристики}

Избраните алгоритми за сравнение са подбрани от Toptal\cite{toptal}. Този сайт не е в никъв случай решаващо място за най-известни алгоритми, но за целите на този документ е достатъчен. 

Избраните алгоритми са:

\begin{itemize}
\item Shell Sort
\item Merge Sort
\item Heap Sort
\item Quick Sort
\end{itemize}

Данни отностно характеристиките им ще бъдат взети от Toptal\cite{toptal} и от warp.povusers.org\cite{comparing_algorithms}. Отново, тези сайтове не са решаващи източници на която и да е от данните, но са достатъчни. 

Избрани характеристики:

\begin{itemize}
\item Сложност
\item Скорост
\item Брой сравнения
\item Брой записвания
\end{itemize}

\newpage

\section{Представителна извадка}

Разглеждайки дадените източници, можем да извадим следните стойности:
\begin{center}
\begin{tabular}{|c|c c c c|}
	\hline
	Име на алгоритъм & Сложност & Скорост & Брой сравнения & Брой записвания \\[3pt]
	\hline\hline
	Shell Sort & 6666 & 0.71 & 101319 & 136506 \\[3pt]
	Merge Sort & 18494 & 0.67 & 56823 & 70000 \\[3pt]
	Heap Sort & 18494 & 0.59 & 107688 & 171318 \\[3pt]
	Quick Sort & 18494 & 0.60 & 95321 & 42888 \\[3pt]
	\hline
\end{tabular}
\end{center}

Важни бележки отностно вземането на тези данни: 
\begin{itemize}
\item сложността е изчислена при горната граница (O) с 5000 елемента
\item скоростта, броят сравнения и броят записвания са взети при сравнение на 5000 случайно разбъркани целочислени числа с минимални повторения помежду им\cite{comparing_algorithms}
\item скоростта (както записано в warp.povusers.org\cite{comparing_algorithms}) е в милисекунди
\item в ситуации на неизвестна или променлива абсолютна максимална сложност, се използва тази сложност, която дава най-висока стойност
\end{itemize}

\section{Статистика на данните}

Снабдени с данни, вече можем да правим статистичен анализ върху тях. Следващите подраздели показват с таближи и диаграми различни свойства и суотношения на тази информация.

\subsection{Вариационни редове}

\begin{center}
\begin{tabular}{|c|c c c c|}
	\hline
	 & Shell sort & Merge Sort & Heap Sort & Quick Sort \\[3pt]
	Сложност & 6666 & 18494 & 18494 & 18494 \\[3pt]
	\hline\hline
	 & Heap sort & Quick Sort & Merge Sort & Shell Sort \\[3pt]
	Скорост & 0.59 & 0.60 & 0.67 & 0.71 \\[3pt]
	\hline\hline
	 & Merge sort & Quick Sort & Shell Sort & Heap Sort \\[3pt]
	Брой сравнения & 56823 & 95321 & 101319 & 107688 \\[3pt]
	\hline\hline
	 & Quick sort & Merge Sort & Shell Sort & Heap Sort \\[3pt]
	Брой записвания & 42888 & 70000 & 136506 & 171318 \\[3pt]
	\hline
\end{tabular}
\end{center}

\subsection{Мода, медиана, средна стойност}

\begin{center}
\begin{tabular}{|c|c c c|}
	\hline
	Характеристика & Мода & Медиана  & Средна стойност\\[3pt]
	\hline\hline
	Сложност & 18494 & $(18494 + 18949)/2 = $\underline{$18494$} & 15537 \\[3pt]
	Скорост & Няма & $(0.60+0.67)/2 = $\underline{$0.635$} & 0.6425 \\[3pt]
	Брой сравнения & Няма & $(95321 + 101319)/2 = $\underline{$98320$} & 90287.75 \\[3pt]
	Брой записвания & Няма & $(70000 + 136506)/2 = $\underline{$103253$} & 105178 \\[3pt]
	\hline
\end{tabular}
\end{center}

\subsection{Сравнение на данни}

\begin{center}
\begin{tikzpicture}  
  
\begin{axis}  
[  
    ybar,  
    enlargelimits=0.15,  
    ylabel={\ Сложност},
    xlabel={\ Алгоритъм},  
    symbolic x coords={Shell, Merge, Heap, Quick}, 
    xtick=data,  
     nodes near coords, % this command is used to mention the y-axis points on the top of the particular bar.  
    nodes near coords align={vertical},  
    ]  
\addplot coordinates {(Shell,6666) (Merge,18494) (Heap,18494) (Quick,18494) };  
  
\end{axis}  
\end{tikzpicture}
 
\textbf{Диаграма 1: Сравнение на сложност}

\begin{tikzpicture}  
\begin{axis}  
[
    ybar,  
    enlargelimits=0.15,  
    ylabel={\ Скорост (в милисекунди)},
    xlabel={\ Алгоритъм},  
    symbolic x coords={Shell, Merge, Heap, Quick}, 
    xtick=data,  
     nodes near coords, % this command is used to mention the y-axis points on the top of the particular bar.  
    nodes near coords align={vertical},  
    ]  
\addplot coordinates {(Shell,0.71) (Merge,0.67) (Heap,0.59) (Quick,0.60) };  
  
\end{axis}  
\end{tikzpicture}  

\textbf{Диаграма 2: Сравнение на скорост}

\begin{tikzpicture}  
\begin{axis}  
[  
    ybar,  
    enlargelimits=0.15,  
    ylabel={\ Брой сравнения},
    xlabel={\ Алгоритъм},  
    symbolic x coords={Shell, Merge, Heap, Quick}, 
    xtick=data,  
     nodes near coords, % this command is used to mention the y-axis points on the top of the particular bar.  
    nodes near coords align={vertical},  
    ]  
\addplot coordinates {(Shell,101319) (Merge,56823) (Heap,107688) (Quick,95321) };  
  
\end{axis}  
\end{tikzpicture}

\textbf{Диаграма 3: Сравнение на брой сравнения}

\begin{tikzpicture}  
\begin{axis}  
[  
    ybar,  
    enlargelimits=0.15,  
    ylabel={\ Брой записвания},
    xlabel={\ Алгоритъм},  
    symbolic x coords={Shell, Merge, Heap, Quick}, 
    xtick=data,  
     nodes near coords, % this command is used to mention the y-axis points on the top of the particular bar.  
    nodes near coords align={vertical},  
    ]  
\addplot coordinates {(Shell,136506) (Merge,70000) (Heap,171318) (Quick,42888) };

\end{axis}
\end{tikzpicture}

\textbf{Диаграма 4: Сравнение на брой записвания}
  
\end{center}

\section{Заключение и анализ на резултатите}

Получената информация дава интересен, но дълбоко непълен, поглед към ефикасността на тези четири алгоритъма. 
Разбира се, данните са събрани при само един тест в само една ситуация, затова каквото заключение да направим, то не бива да бъде взето за дадено.

Quick Sort, макар и думата "бърз" да фигурира в името му, не е толкова по-скоростен от другите алгоритми, даже е по-бавен от Heap Sort!
Като сме започнали да говорим за Heap Sort, той има същиата сложност, обаче му трябват повече сравнения и записвания.

Интересно наблюдение е как сложността на Shell Sort е почти три пъти по-малка от другите три алгоритъма, обаче е почти най-бавния и изискващия най-много сравнения и записвания.

Обаче, в края на деня, се забелязва най-важното нещо: всеки алгоритъм има своя роля. Най-прост е Shell, най-бърз е Heap, най-малко сравнения ползва Merge и най-малко записвания прави Quick. 
Правилния начин на избор на един от тези алгоритми е анализирането на ситуацията в която ще се ползва. 
Но, за тези, които имат три дена да довършат програмата си, Quick Sort е добър избор. 
Във всички категории, които разгледахме, той е на второ или първо място, нещо, което не се вижда често.

\newpage

\nocite{*}
\bibliography{sources}

\end{document}
